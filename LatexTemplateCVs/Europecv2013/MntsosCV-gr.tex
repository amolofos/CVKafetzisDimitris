\documentclass[arial, greek, nologo, notitle, totpages]{europecv2013}
\usepackage{graphicx}
\usepackage[a4paper, top=1.2cm, left=1.2cm, right=1.2cm, bottom=2.5cm]{geometry}
\usepackage[greek, english]{babel}
\usepackage{url}
\usepackage[T1]{fontenc}


%%%%%%%%%%%%%%%%%
%%% Προσωπικά δεδομένα %%%
%%%%%%%%%%%%%%%%%
\ecvname{Καφετζής, Δημήτριος Ανδρέας}
\ecvfootername{Δημήτριος Ανδρέας Καφετζής}
\ecvaddress{\foreignlanguage{english}{xxxx}}
\ecvtelephone[xxxx]{xxxx}
\ecvemail{\href{mailto:kafetzis.dimitris.andreas@gmail.com}{\foreignlanguage{english}{kafetzis.dimitris.andreas@gmail.com}}}
\ecvlinkedin{\href{https://www.linkedin.com/in/kafetzisd}{\foreignlanguage{english}{kafetzisd}}}
\ecvgithub{\href{https://github.com/amolofos}{\foreignlanguage{english}{amolofos}}}
\ecvnationality{Ελληνική}
\ecvdateofbirth{1985-12-20}
\ecvgender{Άρρεν}
\ecvfootnote{Για περισσότερες πληροφορίες επισκεφθείτε την ιστοσελίδα \foreignlanguage{english}{\textcopyright European Union, 2002-2015 | \url{http://europass.cedefop.europa.eu}}}

\begin{document}

\selectlanguage{greek}

\begin{europecv}

\ecvpersonalinfo[10pt]

\ecvposition{Θέση ενδιαφέροντος}{Μηχανικός λογισμικού}

%%%%%%%%%%%%%%%%%
%%% Εργασιακή εμπειρία %%%
%%%%%%%%%%%%%%%%%
\ecvsection{Εργασιακή εμπειρία}

% τα παρακάτω επαναλαμβάνονται για κάθε μία καταχώρηση, ξεκινώντας από την πιο πρόσφατη
\ecvworkexperience{2016-07 - Today}
	{\foreignlanguage{english}{Full stack engineer - Senior Software engineer}}
    {\foreignlanguage{english}{\href{http://www.openbet.com}{Openbet Hellas SA} (part of NYX Gaming Group)} - Φραγκοκκλησιάς 7, Μαρούσι 15125, Αθήνα, Ελλάδα}
    {\foreignlanguage{english}{Software house}}
    {
    \begin{small}
    	Κατά τη διάρκεια της περιόδου αυτής, εργαζόμουν στην Σιγκαπούρη αναπτύσοντας
    	Through out this period, the position is located in Singapore involving the development of heavy transactional software systems for Singapore Pools, Singapore's sports betting and lottery provider. Through this role, i was member of more than one internal teams working with a combination of waterfall and agile development process to introduce and migrate Singapore Pools into Openbet platform.
    	\begin{flushleft}
    		\foreignlanguage{english}{\textit{Engineering stack:} Unix/Linux, IBM Informix database, Postgres, Bash, Tcl, Sql, Ansible, Python, Maven, Jenkins, Yum, Kafka.}
    	\end{flushleft}
    	\textit{Engineering areas:}
    	\begin{itemize}[label=\raisebox{0.25ex}{\tiny$\bullet$}, leftmargin=+1.0cm]
			\item Migrating data from a legacy, monolithic system to a new one being service and component based. We followed the ETL approach of which we were responsible for the transformation and load stages.
			\item Dealing with performance issues of distributed database queries.
			\item Supporting the client's system after the initial roll-out.
			\item Defining release process from code to production deploy.
			\item Entry level experience with queuing systems (Kafka).
		\end{itemize}
	\end{small}
    }

\ecvworkexperience{2015-04 - 2016-06}
	{\foreignlanguage{english}{Full stack engineer - Senior Software engineer}}
    {\foreignlanguage{english}{\href{http://www.openbet.com}{Openbet Hellas SA} - a subsidiary of OpenBet Technologies Ltd} -, Φραγκοκκλησιάς 7, Μαρούσι 15125, Αθήνα, Ελλάδα}
    {\foreignlanguage{english}{Software house}}
    {
   	\begin{small}
    	Full time employment as external contractor to \href{http://www.openbet.com}{OpenBet Technologies Ltd}. During this period i was involved in a small team (5-15 members) supporting the maintenance and evolution of the transactional software systems for some of the largest UK and European betting companies.
        \begin{flushleft}
        	\textit{Engineering stack:} Linux, IBM Informix database, Bash, Tcl, Sql, Html, Javascript, CSS, Maven, Jenkins.
        \end{flushleft}
		\textit{Engineering areas:}
		\begin{itemize}[label=\raisebox{0.25ex}{\tiny$\bullet$}, leftmargin=+1.0cm]
    		\item Migrating data of one of our clients out of a monolithic legacy system following the ETL approaching of which we were implementing the extraction stage.
			\item Refining and improving release process.
			\item Integrating with various payment providers.
		\end{itemize}
    \end{small}
	}

\ecvworkexperience{2011-11 - 2012-07}
	{Στρατεύσιμος}
    {Ελληνική Αεροπορία}
    {
	\begin{small}
    	Οδηγός
	\end{small}
    }

\ecvworkexperience{2009-08 - 2011-02}
	{Τηλεφωνητής, Τέστερ, Τεχνικός}
    {\href{http://www.poseidon.gr}{\foreignlanguage{english}{Poseidon Software SA}}, \foreignlanguage{english}{Konstantinoupoleos 114, Peristeri, Greece}}
    {\foreignlanguage{english}{Software house}}
    {
    \begin{small}
    	Υποστήριξη πελατών από τελεφωνικό κέντρο και στο χώρο τους σχετικά με τα προιόντα λογισμικού της εταιρίας. Επιπλέον, η θέση περιελάμβανε την επαλήθευση των διορθώσεων και νέων λειτουργιών. Τέλος, η θέση συμπεριελάμβανε την εγκατάσταση inhouse συστημάτων λογισμικού σε περιβάλλον \foreignlanguage{english}{Windows} και την εκπαίδευση των τελικών χρηστών.
		\begin{flushleft}
			\foreignlanguage{english}{\textit{Engineering stack:} Windows, MSSql server, Proprietary VB6 \& .dot framework.}
        \end{flushleft}
    \end{small}
	}

\ecvworkexperience{2007-12-41 - 2008-12-24}
	{Τηλεφωνητής}
    {Ο.Τ.Ε. ΑΕ, Γούναρη, Πάτρα}
    {Τηλεπικοινωνίες}
    {
    \begin{small}
    	Τηλεφωνητής στο Κέντρο Πληροφοριών Τηλεφωνικού Καταλόγου Ελλάδος
	\end{small}
    }

%%%%%%%%%%%%%%%%%%%
%%% Εκπαίδευση και κατάρτιση %%%
%%%%%%%%%%%%%%%%%%%
\ecvsection{Εκπαίδευση και κατάρτιση}

% τα παρακάτω επαναλαμβάνονται για κάθε μία καταχώρηση, ξεκινώντας από την πιο πρόσφατη

\ecveducation{2003 - Σήμερα}
	{Φοιτητής του τμήματος Ηλεκτρολόγος Μηχανικός και Τεχνολογίας Υπολογιστών (Η.Μ.\& Τ.Υ.)\vfill \foreignlanguage{english}{\tiny \url{http://www.ece.upatras.gr}}}
    {Πολυτεχνική Σχολή Πανεπιστημίου Πατρών}
    {Ανώτατη}
    {}

\ecveducation{2003}
	{Απολυτήριο Ενιαίου Λυκείου}{22 Ενιαίο Λύκειο Αθηνών}
    {Μέση}
    {}

%%%%%%%%%%%%%%%%%%%%%%%
%%% Ατομικές δεξιότητες και ικανότητες %%%
%%%%%%%%%%%%%%%%%%%%%%%
\ecvsection{Ατομικές δεξιότητες και ικανότητες}

\ecvmothertongue[5pt]{\vfill Ελληνικά}
\ecvitem{\large Άλλη (-ες) γλώσσα (-ες)}{}
\ecvlanguageheader{(*)}
% η παρακάτω γραμμή επαναλαμβάνεται για κάθε μία καταχώρηση
\ecvlanguage{Αγγλικά\\ \foreignlanguage{english}{Proficiency Level}}{\ecvCOne}{\ecvCTwo}{\ecvCOne}{\ecvBTwo}{\ecvBTwo}
\ecvlanguagefooter[10pt]{(*)}

\ecvitem{\large Άδεια οδήγησης}{Κατηγορία Β}

%%%%%%%%%%%%%%%%%%
%%% Πρόσθετες πληροφορίες %%%
%%%%%%%%%%%%%%%%%%
\ecvsection{Πρόσθετες πληροφορίες}

% η παρακάτω γραμμή επαναλαμβάνεται για κάθε μία καταχώρηση
\ecvitem[10pt]{}{\vfill Οι στρατιωτικές μου υποχρεώσεις έχουν εκπληρωθεί (11/2011 - 08/2012).}


\end{europecv}

\end{document}
